\documentclass[12pt]{article}

\usepackage{sbc-template}
\usepackage{graphicx,url}
\usepackage[utf8]{inputenc}
\usepackage[brazil]{babel}
%\usepackage[latin1]{inputenc}

     
\sloppy

\title{Avaliação de desempenho de implementação paralela do método CSEM 3D no supecomputador Santos Dumont}

\author{Mateus F. Lima de Souza\inst{1,2}, Rômulo T. Lima\inst{1,3}}

\address{  Laboratório Nacional de Computação Científica (LNCC)\\
  Getúlio Vargas Av., 333, Quitandinha Petrópolis - RJ - Brasil
\nextinstitute
  Centro Federal de Educação Tecnológica Celso Suckow da Fonseca (CEFET-FR) \\
  R. Gen. Canabarro, 485 - Maracanã, Rio de Janeiro - RJ - Brasil
\nextinstitute
Universidade Católica de Petrópolis (UCP)\\
  R. Barão do Amazonas, 124 - Centro, Petrópolis - RJ - Brasil  
  \email{\{facanha,romulotl\}@lncc.br}
}

\begin{document} 

\maketitle

\begin{abstract}
  This article show the results of a parallel execution at the supercomputer Santos Dumont, the implementation of method CSEM 3D.
\end{abstract}
     
\begin{resumo} 
Este trabalho apresenta resultado de execução paralela no supercomputador Santos Dumont, de implementação do método CSEM 3D. Neste trabalho foram desesnvolvidos conceitos e métodos de implementação de processamento paralelo, operando com grande nível de recursos computacionais. Com isso pudemos através do experimento empirico, constatar que a operação de grande quantidade de recursos, não é um sinônimo de eficiência.
\end{resumo}

\section{Introdução}
Controlled-Source Eletromagnetic (CSEM) é um método de mapeamento geofisico que emprega um monitoramento eletromagnético através de sensores para mapear a resistência elétrica da superfície aquática. Este método é utilizado em larga escala para diversas aplicações, tais como a exploração de hidrocarboneto utilizando tecnologia embarcada. Este trabalho teve como objetivo explorar a eficiência da paralelização MPI do código CSEM 3D~\cite{zerilli2014broadband,zerilli2016broadband}. Como recursos computacionais foram utilizados de 12 até 384 processos MPI no supercomputador Santos Dumont (SDumont).  É apresentada uma análise de desempenho paralelo para dois cenários: utilizando todos os núcleos computacionais e metade deles em cada nó. Percebeu-se uma sensível melhora no desempenho da aplicação com a segunda estratégia.


\section{Metodologia}
\label{sec:metodo}
%
Um diagrama com a arquitetura de CPU multi-core Intel\textregistered~Xeon\textregistered~Gold~6252 disponível entre os nós computacionais Sequana do SDumont é mostrado na Figura~\ref{fig:sequananode}. Cada um destes nós possuem 48 núcleos divididos em 2 sockets contendo 24 núcleos cada. A fim de obter uma análise do comportamento paralelo do código CSEM 3D, foram realizadas execuções paralelas no supercomputador Santos Dumont variando-se de 12 até 384 processos MPI, organizadas e configuradas conforme mostrado na Tabela~\ref{tab:parallelconfig}. Numa primeira configuração, são utilizados o máximo de 48 núcleos computacionais disponíveis entre os nós computacionais Sequana do SDumont. E em outra configuração, são empregados no máximo 24 núcleos por nó.
%
\begin{figure}
\centering
\includegraphics[width=.66\textwidth]{figures/sequanacpudev.png}
\caption{Arquitetura de um nó computacional Sequana no SDumont.}
\label{fig:sequananode}
\end{figure}
%
\begin{table}
\centering
\footnotesize
\caption{Número total de processos MPI, havendo no máximo 48 ou 24 processos MPI por nó, respectivamente nas segunda e terceira colunas.}
\label{tab:parallelconfig}
\begin{tabular}{crr}
\hline
%	&	Máximo de processos MPI por nó	&		\\
\# nós	&	máximo de	&	máximo de	\\
	&	48 MPI/nó	&	24 MPI/nó	\\
\hline
1	&	12	&	12	\\
1	&	24	&	24	\\
1	&	48	&	 -	\\
2	&	96	&	48	\\
4	&	192	&	96	\\
8	&	384	&	192	\\
16	&	 -	&	384	\\
\hline
\end{tabular}
\end{table}

\section{Resultados}


\begin{figure}[ht]
\centering
\includegraphics[width=.5\textwidth]{figures/perfpernode.png}
\caption{Tempo de processamento paralelo com até 384 processos MPI, usando no máximo 48 processos MPI por nó (linha tracejada), e no máximo 24 processos MPI por nó.}
\label{fig:perfpernode}
\end{figure}


\section{Comentários}


\section*{Agradecimentos}
Os autores agradecem a Petróleo Brasileiro S.A. pelo apoio à pesquisa por meio do Termo de Colaboração número 0050.0121778.22.9. Os autores também agradecem ao Laboratório Nacional de Computação Científica (LNCC/MCTI) por fornecer recursos do supercomputador SDumont, que contribuíram para os resultados da pesquisa relatados neste artigo. \textbf{\url{http://sdumont.lncc.br}}.

\bibliographystyle{sbc}
\bibliography{sbc-template}

\end{document}
