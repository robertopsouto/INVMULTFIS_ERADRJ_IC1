\documentclass[12pt]{article}

\usepackage{sbc-template}
\usepackage{graphicx,url}
\usepackage[utf8]{inputenc}
\usepackage[brazil]{babel}
%\usepackage[latin1]{inputenc}

     
\sloppy

\title{Avaliação de desempenho de implementação paralela do método CSEM 3D no supecomputador Santos Dumont}

\author{Mateus F. Lima de Souza\inst{1,2}, Rômulo T. Lima\inst{1,3}}

\address{  Laboratório Nacional de Computação Científica (LNCC)\\
  Getúlio Vargas Av., 333, Quitandinha Petrópolis - RJ - Brasil
\nextinstitute
  Centro Federal de Educação Tecnológica Celso Suckow da Fonseca (CEFET-FR) \\
  R. Gen. Canabarro, 485 - Maracanã, Rio de Janeiro - RJ - Brasil
\nextinstitute
Universidade Católica de Petrópolis (UCP)\\
  R. Barão do Amazonas, 124 - Centro, Petrópolis - RJ - Brasil  
  \email{\{facanha,romulotl\}@lncc.br}
}

\begin{document} 

\maketitle

\begin{abstract}
  This article show the results of a parallel execution at the supercomputer Santos Dumont, the implementation of method CSEM 3D.
\end{abstract}
     
\begin{resumo} 
  Este trabalho apresenta resultado de execução paralela no supercomputador Santos Dumont, de implementação do método CSEM 3D.
\end{resumo}


\section{Introdução}

\section{CSEM} \label{sec:firstpage}
Controlled-Source Eletromagnetic (CSEM) é um método de mapeamento geofisico empregando um monitoramento eletromagnético através de sensores remotos para mapear a resistência elétrica da superfície aquática, este método utilizado em larga escala por diversas aplicações na área mineral e de mineração(Castillo-Reyes
Octavio et al; Sheard et al.2005; Yang \& Oldenburg 2012), estudos de condutividade de cristais(Castillo-Reyes
Octavio et al; Hördt et al.1992, 2000), caracterização de armazenamento de CO2 (Castillo-Reyes
Octavio et al; Girard et al.2011), projeção de reservatório geotérmico (Castillo-Reyes
Octavio et al; Coppo et al.2016) e exploração de hidrocarboneto utilizando tecnologia embarcada (Castillo-Reyes
Octavio et al; Newman \& Alumbaugh 1997; Eidesmo et al.2002; Avdeev 2005; Constable 2006; Srnka et al.2006; Orange et al.2009; Börner 2010; Constable 2010). 
\section{Processamento Paralelo}

\section{Resultados}

\begin{figure}[ht]
\centering
\includegraphics[width=.8\textwidth]{24 por nó (1).pdf}
\caption{This figure is an example of a figure caption taking more than one line and justified considering margins mentioned in.}
\label{fig:24porno}
\end{figure}



\section{Comentários}



Bibliographic references must be unambiguous and uniform.  We recommend giving
the author names references in brackets, e.g. \cite{knuth:84},
\cite{boulic:91}, and \cite{smith:99}.

The references must be listed using 12 point font size, with 6 points of space
before each reference. The first line of each reference should not be
indented, while the subsequent should be indented by 0.5 cm.

\bibliographystyle{sbc}
\bibliography{sbc-template}

\end{document}
